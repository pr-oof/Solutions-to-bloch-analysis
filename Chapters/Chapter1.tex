\newpage
\section*{Chapter 1}

\noindent \textbf{Exercise 1.3.3} Complete the proof for Lemma 1.3.4. That is, prove that \(\cdot\) and \(-\) for \(\integers\) are well-defined. The proof for \(\cdot\) is a bit more complicated
than might be expected.
\begin{proof}
    Let \((a,b), (c,d), (x,y), (z,w) \in \naturals \times \naturals\) be arbitrary. Suppose that \([(a,b)] = [(x,y)]\) and \([(c,d)] = [(z,w)]\). Then \(a + y = b+x\), and since addition on \(\naturals\) is commutative, \(y+a = x+b\).
    Hence \([(y,x)] = [(b,a)]\) which is the same as \(-[(a,b)] = -[(x,y)]\).
    \par Now we show that \(\cdot\) is well-defined. By definition, we have that
    \begin{gather*}
        x+b = y+a \\
        z+d = w + c.
    \end{gather*}
    Our goal is to show that \[(xz + yw) + (bc + ad) = (yz + xw) + (ac + bd)\] which is equivalent to \[[(x,y)] \cdot [(z,w)] = [(a,b)] \cdot [(c,d)].\]
    Consider the following.
    \begin{align*}
        x+b = y+a\\
        (x+b)c = (y+a)c &\qquad \text{Multiply both sides by }c\\
        yw + (x+b)c = yw + yc+ac &\qquad \text{Add }yw\text{ to both sides} \\
        yw + (x+b)c = y(w+c) + ac &\qquad \text{Factor out }y\\
        yw + (x+b)c = y(z+d)+ac &\qquad \text{Substitute }z+a\text{ in place of }w+c\\
        yw + xc + bc + ad = yz + yd + ac + ad &\qquad \text{Distribute and add } ad\text{ to both sides}\\
        yw + xc + bc + ad = yz + ac + (y + a)d &\qquad \text{Factor out }d\\
        yw + xc + bc + ad = yz + ac + (x+b)d &\qquad \text{Substitute in }x+b\text{ in place of }y+a\\
        yw + xc + bc + ad + xw = yz + ac + xd + bd + xw &\qquad \text{Distribute and add }xw\text{ to both sides}\\
        yw + x(w+c) + bc + ad = yz + ac + xd + bd + xw &\qquad \text{Factor out }x\\
        yw + xz+xd + bc + ad = yz + ac + xd + bd + xw &\qquad \text{Substitute }z+d \text{ in place of }w+c \text{ and distribute}\\
        yw + xz + bc + ad = yz + ac + bd + xw &\qquad \text{Cancel }xd \text{ from both sides}\\
        (xz + yw) + (bc + ad) = (yz + xw) + (ac + bd) &\qquad \text{Rearrange}
    \end{align*}
    Hence, \([(xz+yw, yz + xw)] = [(ac + bd, bc + ad)]\).
\end{proof}
\noindent \textbf{Exercise 1.3.5} Prove Theorem 1.3.5 (1) (3) (4) (5) (6) (7) (8) (10) (11) (13) (14).
Throughout the proofs, we let \(x = [(x_0, x_1)]\), \(y = [(y_0, y_1)]\) and \(z = [(z_0, z_1)]\).
\begin{proof}[proof of (1)]
    \begin{align*}
        (x+y)+z &= [(x_0 + y_0, x_1 + y_1)] + [(z_0, z_1)]\\
        &= [((x_0 + y_0) + z_0, (x_1 + y_1) + z_1)]\\
        &= [(x_0 + (y_0 + z_0), x_1 + (y_1 + z_1))]\\
        &= [(x_0, x_1)] + [(y_0 + z_0, y_1 + z_1)]\\
        &= [(x_0, x_1)] + ([(y_0, y_1)] + [(z_0, z_1)])\\
        &= x + (y+z).
    \end{align*}
\end{proof}
\begin{proof}[Proof of (3)]
    \[x + 0 = [(x_0, x_1)] + [(1,1)] = [(x_0 + 1, x_1 + 1)] = [(x_0, x_1)] = x.\]
    One can justify \([(x_0+1, x_1 + 1)] = [(x_0, x_1)]\) by \((x_0 + 1) + x_1 = (x_1 + 1) + x_0\) (We've shown commutativity and associativity in \(\naturals\)).
\end{proof}
\begin{proof}[Proof of (4)]
    \[x + (-x) = [(x_0, x_1)] + [(x_1, x_0)] = [(x_0+x_1, x_1+x_0)] = [(1,1)] = 0.\]
    \([(x_0+x_1, x_1+x_0)] = [(1,1)]\) because \(x_0 + x_1 + 1 = x_1 + x_0 + 1\) (Apply commutativity, which we've proven on \(+\) in \(\naturals\)).
\end{proof}
\begin{proof}[Proof of (5)]
    \begin{align*}
        (xy)z &= [(x_0y_0+x_1y_1, x_0y_1+x_1y_0)] \cdot [(z_0, z_1)] \\
        &= [(z_0(x_0y_0+x_1y_1) + z_1(x_0y_1 + x_1y_0), z_1(x_0y_0 + x_1y_1) + z_0(x_0y_1+x_1y_0))]\\
        &= [(x_0y_0z_0 + x_1y_1z_0 + x_0y_1z_1 + x_1y_0z_1, x_0y_0z_1 + x_1y_1z_1 + x_0y_1z_0 + x_1y_0z_0)]\\
        &= [(x_0(y_0z_0 + y_1z_1) + x_1(y_1z_0 + y_0z_1), x_0(y_0z_1 + y_1z_0) + x_1(y_1z_1 + y_0z_0))]\\
        &= [(x_0, x_1)] \cdot [(y_0z_0 + y_1z_1, y_1z_0 + y_0z_1)]\\
        &= x \cdot ([(y_0, y_1)] \cdot [(z_0, z_1)])\\
        &= x(yz).
    \end{align*}
    Note that we implicitly use properties we've proven for operations on the natural numbers (associativity and commutative for both addition and multiplication).
\end{proof}
\begin{proof}[Proof of (6)]
    \begin{align*}
        xy &= [(x_0, x_1)] \cdot [(y_0, y_1)]\\
        &= [(x_0y_0 + x_1 y_1 , x_0y_1 + x_1 y_0)]\\
        &= [(y_0x_0 + y_1 x_1, y_0x_1 + y_1x_0)]\\
        &= [(y_0, y_1)] \cdot [(x_0, x_1)]\\
        &= yx
    \end{align*}
\end{proof}
\begin{proof}[Proof of (7)]
    \begin{align*}
        x \cdot 1 &= [(x_0, x_1)] \cdot [(1+1, 1)]\\
        &= [(x_0(1+1) + x_1 \cdot 1, x_0 \cdot 1 + x_1(1+1))] \\
        &= [(x_0 + x_0 + x_1, x_0 + x_1 + x_1)] \\
        &= [(x_0, x_1)].
    \end{align*}
    Since \((x_0+x_0+x_1) + x_1 = (x_0 +x_1 + x_1)+x_0\), the final step is justified.
\end{proof}
\begin{proof}[proof of (8)]
    \begin{align*}
        x(y+z) &= [(x_0, x_1)] \cdot ([(y_0+ z_0, y_1 + z_1)])\\
        &= [(x_0(y_0 + z_0) + x_1(y_1 + z_1), x_0(y_1 + z_1) + x_1(y_0 + z_0))]\\
        &= [(x_0y_0 + x_0z_0 + x_1 y_1 + x_1 z_1, x_0 y_1 + x_0 z_1 + x_1 y_0 + x_1 z_0)]\\
        &= [((x_0y_0 + x_1 y_1) + (x_0 z_0 + x_1 z_1), (x_0 y_1 + x_1 y_0) + (x_0 z_1 + x_1 z_0))] \\
        &= [(x_0 y_0 + x_1 y_1, x_0 y_1 + x_1 y_0)] + [(x_0 z_0 + x_1 z_1, x_0 z_1 + x_1 z_0)] \\
        &= [(x_0, x_1)] \cdot [(y_0, y_1)] + [(x_0, x_1)] \cdot [(z_0, z_1)] \\
        &= xy + xz.
    \end{align*}
\end{proof}
\begin{proof}[proof of (10)]
    We show that at least one of \(x < y\), \(x = y\) or \(x > y\) holds. Suppose that \(x \ngtr y\), \(x \nless y\) and \(x \neq y\).
    Hence, \(x_0 + y_1 \nless x_1 + y_0\) and \(y_0 + x_1 \nless y_1 + x_0\). Using the trichotomy of order in \(\naturals\), we can then deduce that \(x_0 + y_1 = x_1 + y_0\).
    However, this contradicts the assumption that \(x \neq y\). Hence, at least one of the three statements holds. Now we show that no two statements can hold simultaneously.
    Suppose that \(x < y\) and \(x = y\). Then \(x_0 + y_1 < x_1 + y_0\) but also \(x_0 + y_1 = x_1 + y_0\) which clearly contradicts the trichotomy of order in \(\naturals\). The other cases follow suit similarly.
\end{proof}
\begin{proof}[proof of (11)]
    Suppose that \(x < y\) and \(y < z\). So \(x_0 + y_1 < x_1 + y_0\) and \(y_0 + z_1 < y_1 + z_0\). From the latter, we deduce that \(y_1 + z_0 = y_0 + z_1 + q\) for some \(q \in \naturals\).
    We can add this equality to both sides of the former to get \((x_0 + y_1) + (y_0 + z_1 + q) < (x_1 + y_0) + (y_1 + z_0)\). Then, using the cancellation law, we can simplify this to
    \(x_0 + z_1 + q < x_1 + z_0\). So \(x_1 + z_0 = x_0 + z_1 + q + r\) for some \(r \in \naturals\). Since \(q+r \in \naturals\), we get that \(x_0 + z_1 < x_1 + z_0\). Hence, \(x < z\).
\end{proof}
\begin{proof}[proof of (13)]
    Suppose that \(x < y\) and \(z > \hat{0}\). We know, from Theorem 1.3.7 (2), that \(z = [(a+1, 1)]\) for some \(a \in \naturals\).
    Since \(x_0 + y_1 < x_1 + y_0\), multiplying both sides by \(a\) yields \(ax_0 + ay_1 < ax_1 + ay_0\).
    Hence,
    \begin{gather*}
        [(ax_0, ax_1)] < [(ay_0, ay_1)] \\
        [(ax_0 + x_0 + x_1, ax_1 + x_0 + x_1)] < [(ay_0 + y_0 + y_1, ay_1 + y_0 + y_1)]\\
        [(x_0(a+1)+x_1, x_1(a+1) + x_0)] < [(y_0(a+1) + y_1, y_1(a+1) + y_0)] \\
        [(x_0, x_1)][(a+1, 1)] < [(y_0, y_1)][(a+1, 1)]\\
        xz < yz
    \end{gather*}
\end{proof}
\begin{proof}[proof of (14)]
    Suppose that \(\hat{0} = \hat{1}\). So \([(1,1)] = [(1+1, 1)]\). By definition, this means \(1 + 1 = 1 + (1+1)\). Applying associativity with the law of cancellation for addition, we get \(1 = 1+ 1\). This contradicts peano axioms. Hence, \(\hat{1} \neq \hat{0}\).
\end{proof}


\newpage

\noindent \textbf{Exercise 1.3.6} Prove Theorem 1.3.7 (1) (3) (4b) (4c).
\begin{proof}[proof of (1)]
    Suppose that \(i(a) = i(b)\) for some \(a,b \in \naturals\). So \([(a+1,1)] = [(b+1, 1)]\) and \((a+1) + 1 = 1 + (b+1)\). Applying the cancellation law twice, we yield \(a=b\). As desired.
\end{proof}
\begin{proof}[proof of (3)]
    This holds true by definition of \(\hat{1}\).
\end{proof}
\begin{proof}[proof of (4b)]
    Let \(a,b \in \naturals\) be arbitrary natural numbers. Then
    \begin{align*}
        i(a)i(b) &= [(a+1, 1)][(b+1, 1)]\\
        &= [((a+1)(b+1) + 1, a+1 + b+1)]\\
        &= [(ab + a + b + 1 + 1, a+1 +  b + 1)]\\
        &= [(ab + 1, 1)] \\
        &= i(ab)
    \end{align*}
\end{proof}
\begin{proof}[proof of (4c)]
    (\(\Longrightarrow\)) Suppose that \(a<b\) for some \(a,b \in \naturals\). Using properties of order, we see that \((a+1) +1 < (b+1) + 1\).
    Therefore, \([(a+1, 1)] < [(b+1, 1)]\).
    \par (\(\Longleftarrow\)) Every step in \(\Longrightarrow\) can be reversed to yield \(\Longleftarrow\).
\end{proof}


\noindent \textbf{Exercise 1.3.7} Let \(x,y,z \in \integers\).
\begin{enumerate}
    \item Prove that \(x < y\) if and only if \(-x > -y\).
    \item Prove that if \(z < 0\), then \(x<y\) if and only if \(xz > yz\).
\end{enumerate}
\begin{proof}[proof for (1)]
    Suppose that \(x<y\). Adding \((-x) + (-y)\) to both sides, we yield \((-y) + ((-x) + x) < (-x) + ((-y) + y)\) which simplifies to \(-y < -x\).
    \par Now suppose that \(-x > -y\). Adding \(x + y\) to both sides gives \(y + (x + (-x)) > x + (y + (-y))\) which gives \(y > x\) after simplification using the law of additive inverses.
\end{proof}
\begin{proof}[proof of (2)]
    Suppose that \(z < 0\), and suppose that \(x < y\). We know from (1) that \(-z > -0 = 0\). And we know from Theorem 1.3.5 (13) that \(-zx < -zy\). Applying (1) again gives \(zx > zy\) which is the desired result.
    \par We prove the converse using constraposition. Suppose that \(x \geq y\). Either \(x = y\) or \(x > y\). In the former, we deduce that \(xz = yz\), and therefore, \(xz \leq yz\). In the latter, we apply the result we just proved to yield
    \(xz < yz\) which is also equivalent to \(xz \leq yz\).
\end{proof}


\noindent \textbf{Exercise 1.3.8} Let \(x \in \integers\). Prove that if \(x > 0\) then \(x \geq 1\).
Prove that if \(x < 0\) then \(x \leq -1\).
\begin{proof}
    Suppose that \(x > 0\), and suppose that \(x < 1\). So \(0 < x < 1\) which contradicts Theorem 1.3.9. Now suppose that \(x < 0\), and that \(x > -1\).
    We deduce that \(-1 < x < 0\) which also contadicts Theorem 1.3.9.
\end{proof}



\noindent \textbf{Exercise 1.3.9}
\begin{enumerate}
    \item Prove that \(1 < 2\).
    \item Let \(x \in \integers\). Prove that \(2x \neq 1\).
\end{enumerate}
\begin{proof}[proof of (1)]
    We know \([(1+1+1, 1)] = 2\) and \([(1+1, 1)] = 1\). Since \((1+1) + 1 < (1+1+1) + 1\), we deduce that \(1 < 2\).
\end{proof}
\begin{proof}[proof of (2)]
    Let \(x \in \integers\) be arbitrary. Suppose that \(2x = 1\). Since \(1 > 0\), we know that \(2x > 0\) (Apply Lemma 1.3.8 (11)). And since \(2 > 0\), \(x > 0\).
    Since \(x\) is positive and \(1 < 2\), it must be the case that \(x < 2x = 1\). Therefore, \(0 < x < 1\). However, we know from Theorem 1.3.9 that no such \(x\) can exist. Therefore, we have a contradiction.
\end{proof}


\newpage
\noindent \textbf{Lemma} If \(A \subseteq \{x \in \integers : x > \hat{0}\}\), \(\hat{1} \in A\), and \(a \in A\) implies \(a + 1 \in A\), then \(A = \{x \in \integers : x > \hat{0}\}\).
\begin{proof}
    Let \(R = \{x \in \integers : x > \hat{0}\}\). Let \(A\) be an arbitrary subset of \(R\) such that \(\hat{1} \in A\). Furthermore, suppose that \(a \in A\) implies \(a+1 \in A\).
    Obviously \(i(1) \in A\). Therefore, \(1 \in i^{-1}[A]\). Now suppose that \(a \in i^{-1}[A]\), meaning \(i(a) \in A\). By the properties of \(A\), we know that \(i(a)+\hat{1} \in A\). Since \(i(a) + \hat{1} = i(a) + i(1) = i(a+1)\), \(i(a+1) \in A\).
    Therefore, \(a+1 \in i^{-1}[A]\). Hence, \(i^{-1}[A] = \naturals\). Therefore, \(R \subseteq A\). Since \(A \subseteq R\), it must be the case that \(A = R\).
\end{proof}


\noindent \textbf{Exercise 1.3.10} Prove that the Well-Ordering Principle (Theorem 1.2.10), which was stated for \(\naturals\) in Section 1.2, still holds when
we think of \(\naturals\) as the set of positive integers. That is, let \(G \subseteq \{x \in \integers : x > 0\}\) be a non-empty set. Prove that there is some \(m \in G\) such that \(m \leq g\) for all \(g \in G\). Use Theorem 1.3.7.
\begin{proof}
    Let \(R = \{x \in \integers : x > 0\}\).
    Suppose that there is no \(m \in G\) such that \(m \leq g\) for all \(g \in G\). We will derive a contradiction.
    Let \[H = \{a \in R :\text{if }b\in R\text{ and }b \leq a\text{, then }b\notin G\}.\]
    It follows from the definition of \(H\) that \(H \cap G = \varnothing\). We will show \(H = R\), using our previous Lemma in the process. It will then follow that \(G\) is empty which gives us our desired contradiction.
    \par Suppose that \(\hat{1} \notin H\). Then there is some \(q \in R\) such that \(q \leq \hat{1}\) and \(q \in G\). Since \(\hat{0} < q < \hat{1}\) contradicts Theorem 1.3.9 and \(\hat{0} < q \leq \hat{1}\), it must be the case that \(q = \hat{1}\).
    Hence, \(\hat{1} \in G\). We know, from Theorem 1.2.9 (2) in \(\naturals\), that \(1 \leq a\) for all \(a \in \naturals\). If we apply \(i : \naturals \to \integers\) to both sides, we get \(\hat{1} \leq i(a)\) for all \(a \in \naturals\). Since \(i[\naturals] = R\), it must be the case that \(\hat{1} \leq r\) for all \(r \in R\).
    But this would mean that \(\hat{1}\) is a least element of \(G\) which is a contradiction to our hypothesis that no such element exists. Therefore, \(\hat{1} \in H\).
    \par Now suppose that \(a \in H\). Suppose further that \(a+\hat{1} \notin H\). Then there is some \(p \in R\) such that \(p \leq a+\hat{1}\) and \(p \in G\). If it were the case that \(p \leq a\), then we would have a contradiction
    due to the fact that \(a \in H\). Hence, by the trichotomy of order in \(\integers\), we see that \(a < p\). Therefore, \(a < p \leq a+\hat{1}\). From which follows immediately that \(p = a+\hat{1}\). Thus, \(a+\hat{1} \in G\). Now let \(x \in G\). Suppose that \(x < a+\hat{1}\). Since \(x\) and \(a\) are elements of \(R\), there exists \(a_0, x_0 \in \naturals\) such that \(i(a_0) = a\) and \(i(x_0) = x\).
    Hence, \(i(x_0) < i(a_0+1)\). Via the properties of \(i :\naturals \to \integers\), we have that \(x_0 < a_0 + 1\). And using Theorem 1.2.9 (10) for \(\naturals\), we see that \(x_0 \leq a_0\). Therefore, \(i(x_0) \leq i(a_0)\) and \(x \leq a\). Because \(a \in H\) it follows that \(x \notin G\), which is a contradiction to the fact no such elements such as \(a+\hat{1}\) exists in \(G\). It follows that \(a+\hat{1} \in H\) and \(H = R\).
\end{proof}

\noindent \textbf{Exercise 1.3.11} Prove Theorem 1.3.8 (1) (3) (4) (5) (7) (10) (11).
\begin{proof}[proof of (1)]
    \begin{gather*}
        x+z = y+z\\
        x + z + (-z) = y + z + (-z) \\
        x + 0 = y + 0\\
        x = y.
    \end{gather*}
\end{proof}
\begin{proof}[proof of (3)]
    Consider \(x + y + (-x) + (-y) = 0\).
    \begin{gather*}
        x+y + (-x) + (-y) = 0\\
        (x+y) + (-(x+y)) + (-x) + (-y) = -(x+y)\\
        0 + (-x) + (-y) = -(x+y)\\
        (-x) + (-y) = -(x+y)
    \end{gather*}
\end{proof}
\begin{proof}[proof of (4)]
    \begin{align*}
        x &= x \cdot 1 \\
        &= x \cdot (1 + 0) \\
        &= x \cdot 1 + x \cdot 0 \\
        &= x + x \cdot 0
    \end{align*}
    So \(x = x + x\cdot 0\). Adding \(-x\) to both sides yields the desired result.
\end{proof}
\begin{proof}[proof of (5)]
    Suppose that \(z \neq 0\) and \(xz = yz\). Then \(xz + (-(yz)) = 0\) and \(xz + (-y)z = (x + (-y))z = 0\).
    Using Theorem 1.3.5 (9) (Which states that \(\integers\) have no zero divisors), we deduce that \(x + (-y) = 0\). Therefore, \(x = y\).
\end{proof}
\begin{proof}[proof of (7)]
    Suppose that \(xy = \hat{1}\). Notice that \(x\) and \(y\) must have the same sign. If they were to have different signs, then (by Lemma 1.3.8 (11)) we'd have that \(xy < 0\) by we know \(1 > 0\). Which leads to a contradiction.
    First, suppose that both \(x\) and \(y\) are positive. We know that there exists \(a,b \in \naturals\) such that \(x = i(a)\) and \(y = i(b)\). So \(i(a)i(b) = i(1)\) and \(i(ab) = i(1)\). Therefore, \(ab = 1\). We know from Theorem 1.2.7 on \(\naturals\) that \(ab = 1\) if and only if \(a = b = 1\). Any other positive solution would lead to a contradiction, therefore, \(x = 1 = y\) are the only positive solutions.
    \par Now suppose that \(x\) and \(y\) are both negative (ie. \(x < 0\) and \(y < 0\)). That means \(-x > 0\) and \(-y > 0\). So there exists \(a,b \in \naturals\) such that \(-x = i(a)\) and \(-y = i(b)\). Since \((-x)(-y) = xy = 1\), we have that \(i(a)i(b) = i(1)\). Using the same argument used when both \(x\) and \(y\) are positive, we have that \(a = 1 = b\) (note that these are the \textbf{only} solutions in \(\naturals\)). Therefore, \(-x = i(1)\) and \(-y = i(1)\). So \(x = -\hat{1}\) and \(y = -\hat{1}\) are the only negative solutions.
\end{proof}
\begin{proof}[proof of (10)]
    Suppose that \(x \leq y\) and \(y \leq x\). Suppose that \(x \neq y\), then \(x < y\) and \(y < x\), which is a clear contradiction to the trichotomy of order in \(\integers\).
\end{proof}
\begin{proof}[proof of (11)]
    Suppose that \(x > 0\) and \(y > 0\). Since \(y > 0\), we have that \(x \cdot y > 0 \cdot y = 0\).
    \par Now suppose that \(x > 0\) but \(y < 0\). Since \(x > 0\), we have that \(y \cdot x < 0 \cdot x = 0\).
\end{proof}


\newpage

\noindent \textbf{Exercise 1.4.1} Prove Lemam 1.4.5 (1) (3) (4) (5) (7) (10)
\begin{proof}[proof of (1)]
    Suppose that \(x + z = y + z\), then \((x + z) + (-z) = (y + z) + (-z)\). Using associativity and the law of additive inverses for addition, we get that \(x + 0 = y + 0\). Using the identity law for addition, we get \(x = y\).
\end{proof}
\begin{proof}[proof of (3)]
    \begin{align*}
        (x+y) + ((-x) + (-y)) &= (y+x) + ((-x) + (-y)) \\
        &= y + (x + ((-x) + (-y))) \\
        &= y + ((x + (-x)) + (-y)) \\
        &= y + (0 + (-y)) \\
        &= y + (-y) \\
        &= 0.
    \end{align*}
    Therefore, \((x+y) + ((-x) + (-y)) = 0\) and adding \(-(x+y)\) to both sides yields the desired result.
\end{proof}
\begin{proof}[proof of (4)]
    \begin{align*}
        x &= x \cdot 1 \\
        &= x \cdot (1 + 0) \\
        &= x \cdot 1 + x \cdot 0.\\
        &= x + x \cdot 0
    \end{align*}
    Adding \(-x\) to both sides yields the desired result.
\end{proof}
\begin{proof}[proof of (5)]
    Suppose that \(xz = yz\) and \(z \neq 0\), then \(xz + (-yz) = 0\). And using Lemma 1.4.5 (6), \(xz + (-y)z\). Factoring out \(z\), we obtain \((x+(-y))z = 0\). Since \(z \neq 0\), it follows (from the fact that \(\integers\) is an integral domain) that \(x+ (-y) = 0\). Adding \(y\) to both sides yields the desired result.
\end{proof}
\begin{proof}[proof of (7)]
    Suppose that \(x > 0\) but \(y < 0\). Since \(x > 0\), Using properties of an ordered integral domain, we have that \(y \cdot x < 0 \cdot x = 0\). Therefore, \(xy < 0\) but \(1 \nless 0\). Assuming \(x < 0\) but \(y > 0\) leads to a similar contradiction.
    Hence, it must be the case that \(y > 0\) and \(x > 0\) simultaneously, or \(y < 0\) and \(x < 0\) simultaneously. (This follows from trichotomy, note that \(x = y = 0\) trivially contradicts \(xy = 1\)).
    \par Suppose that \(xy = 1\). Furthermore, suppose that \(x\) and \(y\) are both greater than \(0\), and that \(x \neq 1\) (WLOG). Using the trichotomy of order, either \(x < 1\) or \(x > 1\). Suppose the former, then \(0 < x < 0 +1\) which is a clear contradiction to Theorem 1.4.6.
    Hence, \(x > 1\). So \(xy > y\) and \(1 > y\). Since \(<\) is transitive, \(0 < y < 0 + 1\). This also contradicts Theorem 1.4.6. Hence, it must be the case that our original assumption is false. That is, \(x \neq 1\) is false and \(x = 1\). A similar argument shows that \(y = 1\).
    \par A similar argument shows that \(x = -1 = y\) are the only negative solutions.
\end{proof}
\begin{proof}[proof of (10)]
    Suppose that \(x \leq y\) and \(y \leq x\). Furthermore, suppose that \(x \neq y\). So \(x < y\) and \(y < x\). This clearly contradicts trichotomy. Hence, \(x = y\).
\end{proof}


\noindent \textbf{Exercise 1.4.2} Let \(n \in \naturals\). Prove that \(n+1 \in \naturals\).
\begin{proof}
    Suppose that \(n \in \naturals\). By definition, that means \(n > 0\). Therefore, \(n+1 > 0 + 1 = 1\). Since \(0 < 1\), we can apply transitivity to show that \(0 < n+1\). Hence, \(n+1 \in \naturals\).
\end{proof}

\newpage
\noindent \textbf{Exercise 1.4.3} Let \(x,y \in \integers\). Prove that \(x \leq y\) if and only if \(-x \geq -y\).
\begin{proof}
    Suppose that \(x \leq y\). Either \(x < y\) or \(x = y\). Suppose the latter. It must follow then that \((-1) \cdot x = (-1) \cdot y\) which is equivalent to \(1 \cdot (-x) = 1 \cdot (-y)\). Therefore, \(-x = -y\) and \(-x \geq -y\).
    Suppose the former. If we add \((-x) + (-y)\) to both sides, we yield
    \begin{gather*}
        x + ((-x) + (-y)) < y + ((-x) + (-y))\\
        (x + (-x)) + (-y) < y + ((-y) + (-x))\\
        0 + (-y) < (y + (-y)) + (-x)\\
        -y < 0 + (-x)\\
        -y < -x
    \end{gather*}
    Therefore, \(-x \geq -y\).
\end{proof}


\noindent \textbf{Exercise 1.4.4} Prove that \(\naturals = \{x \in \integers : x \geq 1\}\).
\begin{proof}
    It suffices to show that \(x > 0\) if and only if \(x \geq 1\). Suppose that \(x > 0\) (\(\Longrightarrow\)).
\end{proof}