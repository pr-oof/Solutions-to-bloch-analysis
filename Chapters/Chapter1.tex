\newpage
\noindent \textbf{Exercise 1.3.5} Prove Theorem 1.3.5 (1) (3) (4) (5) (6) (7) (8) (10) (11) (13) (14).
Throughout the proofs, we let \(x = [(x_0, x_1)]\), \(y = [(y_0, y_1)]\) and \(z = [(z_0, z_1)]\).
\begin{proof}[proof of (1)]
    \begin{align*}
        (x+y)+z &= [(x_0 + y_0, x_1 + y_1)] + [(z_0, z_1)]\\
        &= [((x_0 + y_0) + z_0, (x_1 + y_1) + z_1)]\\
        &= [(x_0 + (y_0 + z_0), x_1 + (y_1 + z_1))]\\
        &= [(x_0, x_1)] + [(y_0 + z_0, y_1 + z_1)]\\
        &= [(x_0, x_1)] + ([(y_0, y_1)] + [(z_0, z_1)])\\
        &= x + (y+z).
    \end{align*}
\end{proof}
\begin{proof}[Proof of (3)]
    \[x + 0 = [(x_0, x_1)] + [(1,1)] = [(x_0 + 1, x_1 + 1)] = [(x_0, x_1)] = x.\]
    One can justify \([(x_0+1, x_1 + 1)] = [(x_0, x_1)]\) using the definition of integers followed by a simple application of the cancellation law.
\end{proof}
\begin{proof}[Proof of (4)]
    \[x + (-x) = [(x_0, x_1)] + [(x_1, x_0)] = [(x_0+x_1, x_1+x_0)] = [(1,1)] = 0.\]
    \([(x_0+x_1, x_1+x_0)] = [(1,1)]\) because \(x_0 + x_1 + 1 = x_1 + x_0 + 1\) (Apply commutativity).
\end{proof}
\begin{proof}[Proof of (5)]
    \begin{align*}
        (xy)z &= [(x_0y_0+x_1y_1, x_0y_1+x_1y_0)] \cdot [(z_0, z_1)] \\
        &= [(z_0(x_0y_0+x_1y_1) + z_1(x_0y_1 + x_1y_0), z_1(x_0y_0 + x_1y_1) + z_0(x_0y_1+x_1y_0))]\\
        &= [(x_0y_0z_0 + x_1y_1z_0z_0 + x_0y_1z_1 + x_1y_0z_1, x_0y_0z_1 + x_1y_1z_1 + x_0y_1z_0 + x_1y_0z_0)]\\
        &= [(x_0(y_0z_0 + y_1z_1) + x_1(y_1z_0 + y_0z_1), x_0(y_0z_1 + y_1z_0) + x_1(y_1z_1 + y_0z_0))]\\
        &= [(x_0, x_1)] \cdot [(y_0z_0 + y_1z_1, y_1z_0 + y_0z_1)]\\
        &= x \cdot ([(y_0, y_1)] \cdot [(z_0, z_1)])\\
        &= x(yz).
    \end{align*}
    Note that we implicitly use properties we've proven for operations on the natural numbers (associativity and commutative for both addition and multiplication).
\end{proof}
\begin{proof}[Proof of (6)]
    \begin{align*}
        xy &= [(x_0, x_1)] \cdot [(y_0, y_1)]\\
        &= [(x_0y_0 + x_1 y_1 , x_0y_1 + x_1 y_0)]\\
        &= [(y_0x_0 + y_1 x_1, y_0x_1 + y_1x_0)]\\
        &= [(y_0, y_1)] \cdot [(x_0, x_1)]\\
        &= yx
    \end{align*}
\end{proof}
\begin{proof}[Proof of (7)]
    \begin{align*}
        x \cdot 1 &= [(x_0, x_1)] \cdot [(1+1, 1)]\\
        &= [(x_0(1+1) + x_1 \cdot 1, x_0 \cdot 1 + x_1(1+1))] \\
        &= [(x_0 + x_0 + x_1, x_0 + x_1 + x_1)] \\
        &= [(x_0, x_1)].
    \end{align*}
    Since \((x_0+x_0+x_1) + x_1 = (x_0 +x_1 + x_1)+x_0\), the final step is justified.
\end{proof}
\begin{proof}[proof of (8)]
    \begin{align*}
        x(y+z) &= [(x_0, x_1)] \cdot ([(y_0+ z_0, y_1 + z_1)])\\
        &= [(x_0(y_0 + z_0) + x_1(y_1 + z_1), x_0(y_1 + z_1) + x_1(y_0 + z_0))]\\
        &= [(x_0y_0 + x_0z_0 + x_1 y_1 + x_1 z_1, x_0 y_1 + x_0 z_1 + x_1 y_0 + x_1 z_0)]\\
        &= [((x_0y_0 + x_1 y_1) + (x_0 z_0 + x_1 z_1), (x_0 y_1 + x_1 y_0) + (x_0 z_1 + x_1 z_0))] \\
        &= [(x_0 y_0 + x_1 y_1, x_0 y_1 + x_1 y_0)] + [(x_0 z_0 + x_1 z_1, x_0 z_1 + x_1 z_0)] \\
        &= [(x_0, x_1)] \cdot [(y_0, y_1)] + [(x_0, x_1)] \cdot [(z_0, z_1)] \\
        &= xy + xz.
    \end{align*}
\end{proof}
\begin{proof}[proof of (10)]
    We show that at least one of \(x < y\), \(x = y\) or \(x > y\) holds. Suppose that \(x \ngtr y\), \(x \nless y\) and \(x \neq y\).
    Hence, \(x_0 + y_1 \nless x_1 + y_0\) and \(y_0 + x_1 \nless y_1 + x_0\). Using the trichotomy of order in \(\naturals\), we can then deduce that \(x_0 + y_1 = x_1 + y_0\).
    However, this contradicts the assumption that \(x \neq y\). Hence, at least one of the three statements holds. Now we show that no two statements can hold simultaneously.
    Suppose that \(x < y\) and \(x = y\). Then \(x_0 + y_1 < x_1 + y_0\) but also \(x_0 + y_1 = x_1 + y_0\) which clearly contradicts the trichotomy of order in \(\naturals\). The other cases follow suit similarly.
\end{proof}
\begin{proof}[proof of (11)]
    Suppose that \(x < y\) and \(y < z\). So \(x_0 + y_1 < x_1 + y_0\) and \(y_0 + z_1 < y_1 + z_0\). From the latter, we deduce that \(y_1 + z_0 = y_0 + z_1 + q\) for some \(q \in \naturals\).
    We can add this equality to both sides of the former to get \((x_0 + y_1) + (y_0 + z_1 + q) < (x_1 + y_0) + (y_1 + z_0)\). Then, using the cancellation law, we can simplify this to
    \(x_0 + z_1 + q < x_1 + z_0\). So \(x_1 + z_0 = x_0 + z_1 + q + r\) for some \(r \in \naturals\). Since \(q+r \in \naturals\), we get that \(x_0 + z_1 < x_1 + z_0\). Hence, \(x < z\).
\end{proof}
\begin{proof}[proof of (13)]
    Suppose that \(x < y\) and that \(z > 0\). So \(z_0 > z_1\). And \(x_0 + y_1 < x_1 + y_0\). Multiplying both sides by \(z_0\), we get \(x_0 z_0 + y_1 z_0 < x_1 z_0 + y_0 z_0\).
    Now consider the following
    \begin{gather*}
        z_1 < z_0\\
        x_0 z_1 < x_0 z_0 \\
        x_0 z_1 + y_1 z_1 < x_0 z_0 + y_1 z_1 \\
    \end{gather*}
\end{proof}
\begin{proof}[proof of (13)]
    Suppose that \(x < y\). Furthermore, suppose that \(z > 0\). Then, by definition, \(z_0 > z_1\) and \(x_0 + y_1 < x_1 + y_0\).
    Assume that \(x_1 > x_0\).
\end{proof}
\begin{proof}[proof of (14)]
    Suppose that \(\hat{0} = \hat{1}\). So \([(1,1)] = [(1+1, 1)]\). By definition, this means \(1 + 1 = 1 + (1+1)\). Applying associativity with the law of cancellation for addition, we get \(1 = 1+ 1\). This contradicts peano axioms. Hence, \(\hat{1} \neq \hat{0}\).
\end{proof}